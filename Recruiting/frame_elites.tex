\begin{frame}[fragile]{Some Words about ``Elite'' PhD Programs}
\begin{itemize}
    \item US News and World Report ranks graduate programs based {\bf solely} on reputation score (voting by department heads)
    \item For example, in 2024 the top 10 for Chemical Engineering are:
    \begin{enumerate}
        \item[1.] MIT
        \item[2.] \textcolor{blue!70!black}{UC Berkeley}, Caltech, Stanford (tied)
        \item[5.] Georgia Institute of Technology
        \item[6.] University of Minnesota
        \item[7.] University of Delaware, Princeton, University of Texas (tied)
        \item[10.] University of Michigan
    \end{enumerate}
    \item Non-US applicants are often overlooked by elite programs \textcolor{green!60!black}{unless}
    \begin{itemize}
        \item A faculty member 
        \begin{itemize}
            \item has direct knowledge of an applicant's institution; and/ or
            \item is from an applicant's country; or
        \end{itemize}
        \item There is history of strong students from an applicant's institution
    \end{itemize}
    \item Elites essentially never waive application fees
    \item Elites often do not recognize credit in graduate courses from other institutions (no MS advantage)
\end{itemize}
\end{frame}

